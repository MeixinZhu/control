\documentclass[final]{article}

% if you need to pass options to natbib, use, e.g.:
% \PassOptionsToPackage{numbers, compress}{natbib}
% before loading nips_2017
%
% to avoid loading the natbib package, add option nonatbib:
% \usepackage[nonatbib]{nips_2017}

\usepackage{nips_2017}
\usepackage{graphicx,url}
\usepackage{amsmath}
%\usepackage{algorithm}
%\usepackage[noend]{algpseudocode}
\usepackage{balance}
\usepackage{gensymb}
\usepackage{multirow}
\usepackage{hhline}
\usepackage{array}
\usepackage[font=small,labelfont=bf]{caption}
\usepackage{dblfloatfix}
\usepackage{gensymb}
\usepackage{movie15}
% to compile a camera-ready version, add the [final] option, e.g.:
% \usepackage[final]{nips_2017}

\usepackage[utf8]{inputenc} % allow utf-8 input
\usepackage[T1]{fontenc}    % use 8-bit T1 fonts
\usepackage{hyperref}       % hyperlinks
\hypersetup{
    colorlinks=true,
    linkcolor=blue,
    filecolor=magenta,      
    urlcolor=blue,
}
\usepackage{url}            % simple URL typesetting
\usepackage{booktabs}       % professional-quality tables
\usepackage{amsfonts}       % blackboard math symbols
\usepackage{nicefrac}       % compact symbols for 1/2, etc.
\usepackage{microtype}      % microtypography

\title{Assignment 1: PID and Model Predictive Control}
% The \author macro works with any number of authors. There are two
% commands used to separate the names and addresses of multiple
% authors: \And and \AND.
%
% Using \And between authors leaves it to LaTeX to determine where to
% break the lines. Using \AND forces a line break at that point. So,
% if LaTeX puts 3 of 4 authors names on the first line, and the last
% on the second line, try using \AND instead of \And before the third
% author name.
%\author{
%  Patrick Lancaster\\
%  Paul G. Allen School of Computer Science and Engineering\\
%  University of Washington\\
%  Seattle WA, 98195 \\
%  \texttt{planc509@cs.washington.edu} \\
  %% examples of more authors
  %% \And
  %% Coauthor \\
  %% Affiliation \\
  %% Address \\
  %% \texttt{email} \\
  %% \AND
  %% Coauthor \\
  %% Affiliation \\
  %% Address \\
  %% \texttt{email} \\
  %% \And
  %% Coauthor \\
  %% Affiliation \\
  %% Address \\
  %% \texttt{email} \\
  %% \And
  %% Coauthor \\
  %% Affiliation \\
  %% Address \\
  %% \texttt{email} \\
%}
\begin{document}
%\nipsfinal
\maketitle
In this assignment, you will implement two methods for controlling the RACECAR. First, you will design a PID controller to control the robot's steering angle as it tries to follow a line (which is represented by a plan consisting of a sequence of poses). Second, you will use model predicitve control to allow a robot to wander around without crashing while simultaneously minimizing its steering angle control effort. Note that in this assignment, many of the implementation details will be left up to you. We encourage you to try out different implementations in order to get the best results.

This assignment can be done as a group. Only one member of the group needs to submit it to the Canvas dropbox.

\section{Getting Started}
Here is some prerequisite information for this assignment:

\begin{enumerate}
\item Please pull the \textit{racecar\_base\_public} repository
\item Before running the updated simulation, install networkx: \textit{sudo easy\_install networkx}
\item Once you run teleop.launch, the simulation will start generating a graph file for the current map. This will take a few minutes to complete. However, the simulation will save the graph file so that it only ever has to be done once per map.
\item While running teleop.launch, you can use the \texttt{2D Pose Estimate} button in the upper ribbon to specify the pose of the robot. You can also use the \texttt{2D Nav Goal} in the upper ribbon to specify a goal pose for the robot. If you have done both of these, the simulation will begin computing a plan from the robot's current pose to the goal pose. This may take awhile depending on the map, but the result can be viewed under the PoseArray topic \textit{/planner\_node/car\_plan}
\item You can change the map that is used by the simulation by editing the 'map' argument of \texttt{racecar\_base\_public/racecar/launch/includes/common/map\_server.launch}. It can reference any of the .yaml files found inside of \texttt{racecar\_base\_public/racecar/maps}
\item Skeleton code for this assignment is available \href{https://gitlab.cs.washington.edu/uw_racecar/course_materials/lab1}{here}.
\end{enumerate}

\section{Line Following With PID Control}
In this section, you will use PID control to steer your robot along a provided plan to reach a goal  pose. It is up to you to define the exact error metric (within reason), as well as the values of the parameters that affect the control policy.

\subsection{In Simulation}
Write a script in \textbf{line\_follower.py} that does the following:

\begin{enumerate}
\item Receives a plan generated by the simulation
\item Subscribes to the current pose of the robot
\item For each received pose
 \begin{enumerate}
 \item Determine which pose in the plan the robot should navigate towards. This pose will be referred to as the target pose.
 \item Compute the error between the target pose and the robot.
 \item Store this error so that it can be used in future calculations (particularly for computing the integral and derivative error)
 \item Compute the steering angle $\delta$ as \\
 
  \centerline{$\delta = k_p*e_t + k_i* \int e_t dt + k_d* \frac{de_t}{dt}$}
 
 \item Execute the computed steering angle
 \end{enumerate}
\end{enumerate}

Also, fill in the launch file \textbf{line\_follower.launch} in order to launch your node. Use this launch file to vary the parameters of your system.

\section{Wandering Around with MPC}

In this section, you will use the kinematic car model to simulate possible trajectories for your robot to follow. Based on received laser scans, your robot will execute a steering angle control that avoids obstacles while minimizing the steering angle control effort.



\section{Following the RACECAR [10 pts]} 

In this section, you will implement the teleoperated car (car A) being followed/lead by a clone (car B). The user will set an offset distance. If this distance is positive, your program should publish a pose for car B that is a fixed distance directly \textit{ahead} of car A. If the distance is negative, the published pose should be a fixed distance directly \textit{behind} car A.

\subsection{In Simulation}

The pose of car A changes as you teleoperate it. Your program should publish the pose of car B as specified above.

\begin{enumerate}

\item Implement a Python script in \textbf{CloneFollower.py} that:
	\begin{itemize}
		\item Subscribes to the pose of car A.
		\item Computes the pose of car B based on where car A is and the given offset distance.
		\item Publishes the pose of car B.
	\end{itemize}
\item Implement a launch file in \textbf{CloneFollower.launch} that:
	\begin{itemize}
		\item Passes an offset distance of 1.5 meters to your script
		\item Launches your script
	\end{itemize}
\item Use your launch file to launch your node and verify that car B stays 1.5 meters directly ahead of car A.

\item Add bounds checking functionality to your script. Use the map to check if the pose computed for Car B is out of bounds. If it is, negate the offset distance (multiply it by negative one), recompute the pose of Car B, and publish the new pose. The result should be that if Car B goes out of bounds while it is \textit{leading} Car A, Car B will start \textit{following} Car A from a fixed distance. If Car B goes out of bounds while \textit{following} Car A, it should start \textit{leading} Car A from a fixed distance.

\item Add a parameter to your launch file that enables/disables bounds checking. 
\item Test that your bounds checking functionality works
\end{enumerate}

%\begin{figure*}[h]
% Always try to place the figure on top of a column to save space
%\centering
% Use linewidth to adjust width
% Use textheight(the total height of the column of text) to adjust height
%\includegraphics[width=\linewidth]{figs/clone_follower.png}
%\framebox[\linewidth]{{\LARGE Explain the entire paper in one figure.}}
%\caption{The pose of Car A in blue, the pose of Car B in green. As Car A is teleoperated, Car B stays 1.5 meters directly in front of Car A.}
%\label{fig:clone_follower}
%\end{figure*}

\section{Controlling the RACECAR [10 pts]} 
In this section, you will use ROS utilities to record control inputs as you drive the robot around. You will then write a program to playback these inputs so that the robot can follow a pre-recorded path. 

\subsection{In Simulation}

\begin{enumerate}
\item Use the \texttt{rosbag} command to record data published to the \textit{/vesc/low\_level/ackermann\_cmd\_mux/input/teleop} topic as you teleoperate the robot to follow a Figure-8 path. The output path of \texttt{rosbag} can be specified with the \texttt{-o} argument.
\item Implement a Python script in \textbf{BagFollower.py} that: 
	\begin{itemize}
    	\item Extracts the data from the recorded bag file
        \item Re-publishes the extracted data such that the robot follows a Figure-8 path without being tele-operated. Publish the data to the \textit{/vesc/high\_level/ackermann\_cmd\_mux/input/nav\_0} topic.
    \end{itemize}
\item Implement a launch file in \textbf{BagFollower.launch} that:
	\begin{itemize}
    	\item Passes the bag file's path to your script
        \item Launches your script
    \end{itemize}
    \item Test your script using the bag file. The robot should move in a Figure-8 pattern without being teleoperated. %Assuming that nothing is currently running on the robot, the following commands should cause your robot to follow a Figure-8 path:\\
    %\centerline{\textbf{roslaunch racecar teleop.launch}}
    %\centerline{\textbf{\textit{\# Wait for robot to finish launching}}}
    %\centerline{\textbf{\textit{\# Hold down the RIGHT bumper on the Logitech controller (it is labeled 'RB')}}}
    %\centerline{\textbf{roslaunch lab0 BagFollower.launch}}
    %\item Submit a video of your robot performing a Figure-8
    \item Add functionality to your script that allows the robot to follow the recorded path backwards (i.e. the robot drives in reverse). Add a parameter to your launch file that specifies whether the robot follows the path forwards or backwards. Test that the robot correctly does a Figure-8 backwards.
\end{enumerate}

\subsection{On Robot}

\begin{enumerate}

\item Collect a bag of you teleoperating the real robot performing a Figure-8
\item Play the bag back so that the robot autonomously does a Figure-8. Assuming that nothing is currently running on the robot, the following commands should cause your robot to follow a Figure-8 path:\\
    \centerline{\textbf{roslaunch racecar teleop.launch}}
    \centerline{\textbf{\textit{\# Wait for robot to finish launching}}}
    \centerline{\textbf{\textit{\# Hold down the RIGHT bumper on the Logitech controller (it is labeled 'RB')}}}
    \centerline{\textbf{roslaunch lab0 BagFollower.launch}}
\item Collect a bag of the robot travelling a path on the Third floor of Sieg hall, similar to the one shown below.
\item Again play this bag back so that the robot autonomously executes the commands recorded in the bag.

\end{enumerate}

%\begin{figure}[h]
%\begin{center}
%\includegraphics[width=0.6\linewidth]{figs/bag_follower.png}
%\end{center}
%\caption{The path of the teleoperated robot is shown in green.}
%\end{figure}

\section{Assignment Submission}

Submit the following:

\begin{enumerate}

\item Video demonstrating your CloneFollower following the teleoperated car and detecting when it goes out of bounds. (You can use Kazam to take a video of your screen.)

\item Video of your RACECAR performing a figure-8 in simulation

\item The two Python scripts and two launch files that you wrote.
\end{enumerate}

In addition, if you have chosen to do the real robot track, submit the following:
\begin{enumerate}
\item Video of the real robot car performing a figure-8.
\item Video of the real robot at least attempting to follow the specified path in Sieg Hall.
\item Answers to the following questions:
	\begin{itemize}
		\item You should have collected two bags of the robot doing a figure-8, one in simulation, and one on the real robot. Play back both of these bags on the real robot. Explain any significant differences between the two performances.
		\item Is the robot successfully able to navigate through Sieg Hall when playing back the corresponding bag? Note any deficiencies in performance. Explain why these deficiencies might occur.
	\end{itemize}

\end{enumerate}

\end{document}

